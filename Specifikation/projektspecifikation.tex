\documentclass[a4paper]{article}

\usepackage[swedish]{babel}
\usepackage[utf8]{inputenc}
\usepackage{graphicx}

\title{Projektspecifikation}

\author
{
	Martin Pola \\ \texttt{<mpola@kth.se>} \and
	Sandra Järkeborn \\ \texttt{<jark@kth.se>}
}

\begin{document}
	\maketitle
	
	Inspirerat av funktionaliteten i Stay A While tänkte vi göra ett eget kösystem. Förutom grundläggande köfunktionalitet tänkte vi adressera flera av de brister vi tycker oss ha sett i dagens Stay A While. Det tar sig i uttryck främst genom ökad funktionalitet och ökade inställningsmöjligheter, för varje enskild kö. Nedan följer en sammanfattning över den funktionalitet vi planerar att implementera. Det är vår intention att projektet ska uppnå betyg A.
	
	\paragraph{Alla ska kunna...}
	\begin{itemize}
		\item logga in med sitt KTH-konto.
		\item se alla köer. För att se namnen på de som står i en kö måste man vara inloggad.
	\end{itemize}
	
	\paragraph{Studenter ska kunna...}
	\begin{itemize}
		\item ställa sig i köer. För att ställa sig i en kö måste man ange sin datorplats, en kommentar och huruvida man vill presentera eller ha hjälp.
		\item markera att de tar emot hjälp.
		\item lämna köer.
		\item få sin datorplats automatiskt ifylld om de är inloggade på en av skolans datorer. Då ska platsfältet inte gå att redigera.
	\end{itemize}

	\paragraph{Assistenter ska kunna...}
	\begin{itemize}
		\item ta bort studenter från kön.
		\item flytta studenter i kön.
		\item markera att studenter tar emot hjälp.
		\item rödmarkera en student för att påvisa ogiltig eller okänd placering.
		\item skicka popupmeddelanden till alla studenter eller alla assistenter.
		\item ändra beskrivning på kön. I beskrivningsfältet ska det gå att ha radbryten och klickbara länkar. När man redigerar en beskrivning ska den nuvarande beskrivningen finnas med som utgångspunkt i textfältet.
		\item stänga och öppna kön.
		\item tömma kön.
		\item välja om kön ska vara begränsad till de som sitter i vissa terminalsalar eller inte. Om sådana begränsningar finns krävs det att man är inloggad på en skoldator som har blivit automatiskt ifylld som plats.
		\item ställa in en tidpunkt för när kön automatiskt ska öppnas.
		\item välja om kön automatiskt ska rensas, och i så fall vid vilken tid på dygnet.
		\item välja att slå på en vitlista, och att i så fall endast studenter som finns på den kan gå med i kön.
	\end{itemize}
	
	Dessutom ska det webbaserade gränssnittet ackompanjeras av en Android-app, genom vilken assistenter kan genomföra grundläggande hantering av kön.
	
	\paragraph{Lärare ska kunna...}
	\begin{itemize}
		\item lägga till och ta bort assistenter i köer.
		\item skapa och ta bort köer.
		\item specificera terminalsalar med tillhörande datorplatser.
	\end{itemize}

\end{document}
